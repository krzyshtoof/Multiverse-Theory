\documentclass[12pt]{article}
\usepackage{amsmath, amssymb, hyperref}

\title{Mass as Information: A Quantum-Informational Perspective}
\author{Krzysztof W\l{}odzimierz Banasiewicz \\\\ Independent Researcher, The Hague / Wroc\l{}aw}
\date{\today}

\begin{document}
\maketitle

\begin{abstract}
We propose that mass can be understood as a manifestation of quantum information.
Building on the equivalence between energy and mass ($E=mc^2$) and the thermodynamic cost
of information processing (Landauer's principle), we argue that mass represents stored
and structured information within quantum fields. We connect this idea with black hole
thermodynamics, entanglement networks, and the Higgs mechanism, suggesting that
in a fundamental sense, \textbf{mass is information}.
\end{abstract}

\section{Introduction}
Mass has traditionally been viewed as a measure of inertia and the source of gravitation.
However, developments in black hole physics, quantum information theory, and quantum gravity
invite a deeper interpretation. If energy is equivalent to information, and mass is equivalent to energy,
then mass itself can be seen as a manifestation of information.

\section{Mass--Energy--Information Equivalence}
Einstein's relation
\begin{equation}
E = mc^2
\end{equation}
combined with Landauer's principle
\begin{equation}
E_{bit} = k_B T \ln 2
\end{equation}
implies that the mass $m$ associated with a system can be viewed as the information content $I$
encoded within it:
\begin{equation}
m \equiv \frac{I \, k_B T \ln 2}{c^2}.
\end{equation}

\section{Black Hole Thermodynamics}
The Bekenstein--Hawking entropy of a black hole of horizon area $A$ is
\begin{equation}
S = \frac{k_B c^3 A}{4 G \hbar}.
\end{equation}
Since entropy is a measure of information, the mass of a black hole is
directly related to the information it contains.
This establishes an explicit connection between mass and information.

\section{Entanglement Networks}
In tensor network models of emergent spacetime, such as MERA and holographic duality,
geometry arises from patterns of entanglement. Within this framework, mass corresponds
to localized constraints or ``nodes'' of information flow in the entanglement graph.
The ER=EPR conjecture suggests that mass and spacetime geometry are manifestations
of quantum information.

\section{Higgs Mechanism as Informational Resistance}
Elementary particles acquire mass via interaction with the Higgs field.
This interaction can be reinterpreted as the informational cost of embedding
a quantum state into the structure of the vacuum, i.e. the ``resistance''
of the vacuum to information flow.

\section{Conclusion}
Mass can be reinterpreted as an emergent manifestation of quantum information.
From black holes to particle physics, evidence accumulates that mass is not
a fundamental property but a measure of structured, encoded information
in quantum reality. This perspective may guide the development of a
quantum theory of gravity where spacetime, matter, and mass emerge from
the same informational substrate.

\end{document}

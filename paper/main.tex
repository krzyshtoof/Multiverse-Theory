\documentclass[11pt,a4paper]{article}
\usepackage[utf8]{inputenc}
\usepackage[T1]{fontenc}
\usepackage{lmodern}
\usepackage{hyperref}
\usepackage{amsmath,amssymb,amsfonts}
\usepackage{graphicx}
\usepackage{authblk}
\usepackage{cite}

\title{Sieć Splątania Kwantowego jako Fundament Multiwersum:\\
Nowy Paradygmat Czasu Relacyjnego i Napędu}

\author[1]{Krzysztof Włodzimierz Banasiewicz}
\affil[1]{Independent Researcher, The Hague, Netherlands}
\date{\today}

\begin{document}
\maketitle

\begin{abstract}
Proponujemy model, w którym globalna sieć splątania kwantowego stanowi substrat Multiwersum, z którego emergują czasoprzestrzeń i świadomość. Wprowadzamy formalizm relacyjnego czasu oraz hipotetyczny „operator mostu” skracający efektywne interwały. Omawiamy implikacje empiryczne i szkic implementacji bramki wielokubitowej jako prototypu „napędu relacyjnego”.
\end{abstract}

\section{Wstęp}
% kontekst, motywacja

\section{Formalizm}
% definicje, notacja, dynamika

\section{Operator Mostu Relacyjnego}
% konstrukcja operatora, warunki, własności

\section{Konsekwencje Empiryczne}
% przewidywania, testowalność

\section{Dyskusja i prace przyszłe}
% ograniczenia, dalsze kroki

\bibliographystyle{ieeetr}
\bibliography{refs}
\end{document}



\documentclass[aps,prd,twocolumn,nofootinbib,superscriptaddress]{revtex4-2}
\usepackage{amsmath,amssymb,graphicx}
\usepackage{hyperref}
\usepackage{xcolor}

\begin{document}

\title{Ψ\_connection: A Topological Framework for the Emergence of Spacetime from Quantum Entanglement}
\author{Krzysztof W. Banasiewicz}
\affiliation{Independent Researcher}
\date{\today}

\begin{abstract}
We introduce the \textit{Ψ\_connection} framework --- a conceptual model describing spacetime as an emergent structure arising from quantum entanglement geometry. The theory postulates that classical spacetime and an atemporal quantum layer coexist, connected through a relational function $\Phi$. This entanglement-driven structure encodes the topology and geometry of the Universe. The model draws from Everettian quantum mechanics, general relativity (white/black holes), and topological methods, proposing a new interpretation of the Big Bang as a collapse of the Universe’s wavefunction. We outline the mathematical architecture and its implications for quantum gravity research.
\end{abstract}

\maketitle

\section{Introduction}
Understanding the nature of spacetime and its connection to quantum mechanics remains a fundamental challenge in theoretical physics. This work proposes the \textit{Ψ\_connection} framework, in which spacetime emerges from a topological structure of quantum entanglement encoded by a function $\Phi$. The theory suggests the Universe consists of two complementary layers: a classical spacetime domain and an atemporal quantum domain, interconnected via entanglement relations. The Big Bang is reinterpreted as a decoherence event of the universal wavefunction $\Psi$.

\section{Theoretical Framework}
The key hypothesis is that the Universe's geometry results from correlations between quantum states. Two main mathematical objects are introduced:
\begin{itemize}
    \item $\Psi(t, \mathbf{r})$: the wavefunction of a "quantum thread".
    \item $\Phi(\mathbf{r}', \mathbf{r})$: the relational function encoding entanglement.
\end{itemize}

These are linked via:
\[
\Psi(t, \mathbf{r}) = \int \Phi(\mathbf{r}', \mathbf{r}) \cdot e^{-iEt/\hbar} \, d\mathbf{r}'
\]

$\Phi$ is interpreted as the topological "wiring" of space, while $\Psi$ encodes temporal evolution.

\section{Mathematical Structure}
$\Phi$ is proposed to function analogously to a propagator, mapping quantum entanglement across the spatial manifold. The underlying space may resemble Calabi-Yau-like compact topologies. Time emerges as a phase term $e^{-iEt/\hbar}$ in the evolution of $\Psi$. Tools from configuration integrals, operator theory, and spin networks are anticipated to build a full formalism.

\section{Relation to Existing Theories}
\begin{itemize}
    \item \textbf{Quantum Gravity}: Compatible with LQG and spin foam frameworks.
    \item \textbf{ER=EPR}: $\Phi$ resembles entanglement maps defining spacetime links.
    \item \textbf{Many-Worlds}: $\Psi$ remains globally coherent; collapse is cosmological.
\end{itemize}

\section{Outlook}
Future work includes:
\begin{enumerate}
    \item Defining operator-based $\Phi$,
    \item Building simulation tools,
    \item Exploring CMB and Planck-scale effects,
    \item Bridging with categorical and algebraic QFT methods.
\end{enumerate}

\section*{Acknowledgments}
This conceptual framework emerged through a synthesis of quantum theory, topology, and cosmological intuition. The author acknowledges the support of collaborative AI tools in drafting and visualizing early ideas.

\bibliographystyle{unsrt}
\begin{thebibliography}{9}
\bibitem{everett}
H. Everett, ``Relative State'' Formulation of Quantum Mechanics, Rev. Mod. Phys. 29, 454 (1957).
\bibitem{maldacena}
J. Maldacena, The Large-N Limit of Superconformal Field Theories and Supergravity, Adv. Theor. Math. Phys. (1998).
\bibitem{penrose}
R. Penrose, \textit{Cycles of Time}, Bodley Head (2010).
\bibitem{rovelli}
C. Rovelli, \textit{Quantum Gravity}, Cambridge University Press (2004).
\bibitem{github}
K. W. Banasiewicz, Multiverse-Theory GitHub Repository, \url{https://github.com/krzyshtoof/Multiverse-Theory}
\end{thebibliography}

\end{document}
